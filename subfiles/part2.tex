%! TEX root = part2.tex
% the main TeX file which is intended to compile, :VimtexReload after adjustment
% :h vimtex-tex-root
\documentclass[../gbr_teknik.tex]{subfiles}
% \includeonlyframes{current}
% \setbeameroption{hide notes} % Only slides
% \setbeameroption{show only notes} % Only notes
% \setbeameroption{show notes on second screen=right} % Both

\begin{document}

\section{Huruf dan Angka}%
\label{sec:Huruf dan Angka}

\begin{frame}[c,mycolor=digiPH_white,mytitle=standard,light]
	\framesubtitle{Pengertian}

	\textbf{Huruf dan angka} dalam gambar teknik digunakan untuk:
	\begin{itemize}
		\item Menunjukkan ukuran
		\item Memberikan keterangan bagian
		\item Menulis catatan pada etiket
		\item Menjelaskan informasi teknis
	\end{itemize}

	Tujuan utama:
	\begin{itemize}
		\item Memperjelas informasi
		\item Menghindari salah tafsir
	\end{itemize}
\end{frame}


\begin{frame}[c,mycolor=digiPH_ocean,mytitle=center,dark]Agar berfungsi dengan baik, \textit{huruf dan angka harus:}

	\begin{itemize}
		\item Jelas
		\item Seragam
		\item Proporsional
		\item Dapat direproduksi (microfilm/fotokopi)
		\item Tidak terlalu kecil
	\end{itemize}\end{frame}


\begin{frame}[c,mycolor=digiPH_white,mytitle=standard,light]\frametitle{Unsur Proporsi Huruf}

	Dalam gambar teknik dikenal parameter:
	\begin{itemize}
		\item h = tinggi huruf besar
		\item c = tinggi huruf kecil
		\item a = jarak huruf
		\item b = jarak garis
		\item e = jarak kata
		\item d = tebal huruf
	\end{itemize}

	Semua harus mengikuti standar proporsi.\end{frame}


\begin{frame}[c,mycolor=digiPH_leaf,mytitle=standard,dark]\frametitle{Tipe Huruf Gambar Teknik}

	\framesubtitle{Terdapat dua tipe standar:}

	Tipe A \\
	Tipe B

	Perbedaan utama:

	Tipe A → lebih tipis\\
	Tipe B → lebih tebal dan lebih jelas
\end{frame}


\begin{frame}[c,mycolor=digiPH_ocean,mytitle=center,dark]\frametitle{Bentuk Huruf}

	Huruf standar terdiri dari:
	\begin{itemize}
		\item Huruf tegak (upright)
		\item Huruf miring 15°
	\end{itemize}

	Digunakan secara konsisten dalam satu gambar.\end{frame}

\begin{frame}[c,mycolor=digiPH_white,mytitle=standard,light]

	\includegraphics[width=.6\textwidth]{../figures/bentukhuruf.jpg}

\end{frame}

\section{Konstruksi Geometris}%
\label{sec:Konstruksi Geometris}

\begin{frame}[c,mycolor=digiPH_ocean,mytitle=center,dark]Konstruksi geometris adalah teknik menggambar {\huge bentuk-bentuk dasar} secara presisi menggunakan alat bantu:

	\begin{itemize}
		\item Jangka
		\item Penggaris
		\item Segitiga
		\item Busur derajat
	\end{itemize}

	Digunakan untuk:
	\begin{itemize}
		\item Meningkatkan akurasi
		\item Menghindari perkiraan kasar
	\end{itemize}
\end{frame}

\begin{frame}[c,mycolor=digiPH_leaf,mytitle=standard,dark]\framesubtitle{Konstruksi Dasar yang Sering Digunakan}

	\begin{itemize}
		\item Membagi garis
		\item Garis tegak lurus
		\item Membagi sudut
		\item Membuat segi lima
		\item Membuat segi enam
		\item Membuat ellips
	\end{itemize}
\end{frame}

\begin{frame}[t,mycolor=digiPH_white,mytitle=standard,light]

	\includegraphics<1>[width=.65\textwidth]{garis lurus.jpg}
	\includegraphics<2>[width=.65\textwidth]{garis tegak lurus.jpg}
	\includegraphics<3>[width=.65\textwidth]{bagi sudut.jpg}
	\includegraphics<4>[width=.62\textwidth]{segi lima diketahui.jpg}
	\includegraphics<5>[width=.65\textwidth]{segi lima lingkaran.jpg}
	\includegraphics<6>[width=.68\textwidth]{segienamlingkaran.jpg}
	\includegraphics<7>[width=.65\textwidth]{segienamluarlingkar.jpg}
	\includegraphics<8>[width=.62\textwidth]{elips dua lingkaran.jpg}
	\includegraphics<9>[width=.62\textwidth]{elips segi empat.jpg}

\end{frame}

\begin{frame}[t,mycolor=digiPH_ocean,mytitle=center,dark]

	\tabitem Fungsi Huruf dan Angka
	Digunakan untuk:
	\begin{itemize}
		\item Ukuran
		\item Keterangan bagian
		\item Catatan teknis
	\end{itemize}
	\tabitem Harus:
	\begin{itemize}
		\item Jelas
		\item Seragam
		\item Proporsional
	\end{itemize}

\end{frame}


\begin{frame}[t,mycolor=digiPH_red,mytitle=standard,dark]
	\frametitle{Standar Huruf}

	\includegraphics<1>[width=.9\textwidth]{../figures/bentukhuruf.jpg}
	\includegraphics<2>[width=.9\textwidth]{../figures/bentukhurufmiring.jpg}

	\includegraphics<3>[width=.9\textwidth]{../figures/huruf.jpg}

\end{frame}




\end{document}
