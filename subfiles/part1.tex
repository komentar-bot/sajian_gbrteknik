%! TEX root = part1.tex
% the main TeX file which is intended to compile, :VimtexReload after adjustment
   % this file should be included in the main file
% :h vimtex-tex-root
\documentclass[../gbr_teknik.tex]{subfiles}
%ensure the class of docs

\begin{document}


\begin{frame}[t,mycolor=digiPH_white,mytitle=standard,light]
	\frametitle{Capaian Pembelajaran}

	\begin{itemize}
		\item<-4> Menjelaskan fungsi gambar dalam teknik mesin.
		\item<-4> Membedakan ukuran kertas gambar.
		\item<-4> Membuat format kertas gambar sesuai standar.
		\item<-4> Menggunakan skala dengan benar.
		\item<5-6> Menerapkan standar garis gambar.
		\item<5-6> Menyusun etiket gambar.
		      % \item<5-8> Menggunakan huruf dan angka sesuai standar.
		      % \item<5-8> Membuat konstruksi geometris dasar.
	\end{itemize}
	% \note{dummy text untuk menghindari note pertama tertimpa isi slide}
\end{frame}

\begin{frame}[t,mycolor=digiPH_leaf,mytitle=standard,dark]

	\frametitle{Mengapa Disebut Bahasa Teknik?}

	\begin{itemize}
		\item Media komunikasi visual dalam industri.
		\item Menyampaikan bentuk, ukuran, dan spesifikasi.
		\item Tidak perlu membawa benda asli.
		\item Berlaku lintas daerah dan negara.
	\end{itemize}

	% \noindent\textbf{📌 Gambar = “Bahasa” dalam industri permesinan

\end{frame}


\begin{frame}[t,mycolor=digiPH_ocean,mytitle=center,dark]

	\frametitle{Fungsi Gambar Teknik}
	Sarana komunikasi
	\begin{itemize}
		\item Perencana $\leftrightarrow$ Pelaksana
		\item Manajemen $\leftrightarrow$ Teknisi
	\end{itemize}

	Menuangkan gagasan
	\begin{itemize}
		\item Dari konsep $\rightarrow$ Visual teknis
		\item Dari visual $\rightarrow$ Produk nyata
	\end{itemize}
\end{frame}


\begin{frame}[t,mycolor=digiPH_red,mytitle=standard,dark]
	\frametitle{Syarat Menjadi Bahasa Teknik}
	Agar efektif, gambar harus:
	\begin{itemize}
		\item Mengikuti standar
		\item Menggunakan simbol baku
		\item Menggunakan ukuran dan skala benar
		\item Mudah dibaca dan dipahami
	\end{itemize}

	\noindent Standar = ``Tata bahasa'' gambar teknik

\end{frame}

\begin{frame}[t,mycolor=digiPH_ocean,mytitle=center,dark]

	\frametitle{Manual vs Digital}
	Walaupun sekarang menggunakan:
	\begin{itemize}
		\item AutoCAD
		\item SolidWorks
		\item Software CAD lainnya
	\end{itemize}

\end{frame}

\begin{frame}[t,mycolor=digiPH_red,mytitle=standard,dark]
	\frametitle{Menggambar manual tetap penting untuk:}
	\begin{itemize}
		\item Melatih ketelitian
		\item Memahami prinsip dasar
		\item Melatih koordinasi dan presisi
	\end{itemize}


\end{frame}

\begin{frame}[t,mycolor=digiPH_leaf,mytitle=standard,dark]
	\frametitle{Standar Ukuran Kertas}
	\begin{table}[h]
		\centering
		\caption{Ukuran Kertas Standar}
		\begin{tabular}{|c|c|c|}
			\hline
			Ukuran  &  Lebar (mm)  &  Panjang (mm)  \\
			\hline
			A0      &  841         &  1189  \\
			A1      &  594         &  841  \\
			A2      &  420         &  594  \\
			A3      &  297         &  420  \\
			A4      &  210         &  297  \\
			A5      &  148         &  210  \\
			\hline
		\end{tabular}
	\end{table}


\end{frame}

\begin{frame}[t,mycolor=digiPH_leaf,mytitle=standard,dark]
	\frametitle{Posisi Kertas}

	A0 – A3 → Posisi mendatar (landscape)
	A4 – A6 → Posisi tegak (portrait)

	Harus sesuai standar tata letak margin.


\end{frame}


\begin{frame}[t,mycolor=digiPH_white,mytitle=standard,light]

	\frametitle{Skala Gambar}
	Skala = Perbandingan ukuran gambar dengan ukuran asli.

	Skala Pengecilan

	1:2, 1:5, 1:10, 1:20, 1:50, 1:100, dst.

	Skala Pembesaran

	2:1, 5:1, 10:1


\end{frame}

\begin{frame}[t,mycolor=digiPH_white,mytitle=standard,light]
	\frametitle{Jenis Garis dalam Gambar Teknik}

	\begin{table}[h]
		\centering
		\caption{Jenis Garis dan Fungsinya}
		\begin{tabular}{|l|l|}
			\hline
			Jenis Garis         &  Fungsi  \\
			\hline
			Garis tebal         &  Garis benda  \\
			Garis tipis         &  Garis ukuran, bantu  \\
			Garis putus-putus   &  Bayangan  \\
			Garis titik         &  Sumbu  \\
			Garis bergelombang  &  Potongan  \\
			\hline
		\end{tabular}
	\end{table}
\end{frame}

\begin{frame}[t,mycolor=digiPH_ocean,mytitle=center,dark]

	\textbf{Perbandingan standar:}
	\begin{itemize}
		\item Garis tebal = $s$
		\item Garis tipis = $\frac{1}{4}s$
		\item Garis putus-putus = $\frac{1}{2}s$
	\end{itemize}

	\textbf{Tujuan:}
	\begin{itemize}
		\item Membedakan fungsi garis
		\item Memudahkan pembacaan gambar
	\end{itemize}

\end{frame}

\begin{frame}[t,mycolor=digiPH_red,mytitle=standard,dark]
	Etiket = Identitas gambar kerja
	\begin{itemize}
		\item Judul gambar
		\item Skala
		\item Nama pembuat
		\item Tanggal
		\item Nomor gambar
	\end{itemize}

\end{frame}

\begin{frame}[t,mycolor=digiPH_red,mytitle=standard,dark]
	\frametitle{Tugas Rumah}
	Buatlah Gambar Etiket
\end{frame}

\bibliographystyle{apalike}
\bibliography{../biblio.bib} % required for citecompletion, not work in mainfile
\end{document}

